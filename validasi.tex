\documentclass[12pt,a4paper]{article}

% Required packages
\usepackage[utf8]{inputenc}
\usepackage[T1]{fontenc}
\usepackage{graphicx}
\usepackage{geometry}
\usepackage{fancyhdr}
\usepackage{array}
\usepackage{booktabs}
\usepackage{tabularx}
\usepackage{multirow}

% Page setup
\geometry{
    top=4cm,
    bottom=2cm,
    left=2.5cm,
    right=2.5cm
}

% Header and footer setup
\pagestyle{fancy}
\fancyhf{} % Clear all header and footer fields

% Define the header
\fancyhead[C]{%
    \renewcommand{\arraystretch}{1.0}
    \fbox{%
        \begin{tabular}{@{}p{12.85cm}|p{2.5cm}@{}}
            \multicolumn{1}{c}{\normalsize\textbf{KUNCI JAWABAN DAN PEDOMAN PENSKORAN}} & 
            \multirow{6}{*}{\centering\includegraphics[width=2.5cm,height=2.5cm,keepaspectratio]{itera.png}} \\
            & \\
            {\small\renewcommand{\arraystretch}{0.8}
            \begin{tabular}{@{}p{5cm}@{\ :\ }p{7.4cm}@{}}
                Nama mata kuliah (kode) & \\
                Program Studi & \\
                Dosen pengampu & \\
                Semester (tahun pelajaran) & \\
                Tugas/ujian yang dinilai & \\
            \end{tabular}} & \\
        \end{tabular}%
    }
}

% Remove header rule
\renewcommand{\headrulewidth}{0pt}

% Adjust header height and positioning
\setlength{\headheight}{232.05008pt}
\addtolength{\topmargin}{-61.33356pt}
\setlength{\headsep}{1cm}

\begin{document}

% Question 1 box
\noindent
\renewcommand{\arraystretch}{1.2}
\begin{tabular}{|>{\centering}m{2cm}|p{13cm}|}
\hline
\textbf{1} & \textbf{Pertanyaan:} Lorem ipsum dolor sit amet, consectetur adipiscing elit. Sed do eiusmod tempor incididunt ut labore et dolore magna aliqua. Ut enim ad minim veniam, quis nostrud exercitation ullamco laboris? \\
\hline
\end{tabular}

% Rubric table for question 1
\noindent
\renewcommand{\arraystretch}{1.2}
\begin{tabular}{|p{12cm}|p{3cm}|}
\hline
\textbf{Aspek / Konsep yang dinilai} & \textbf{Skor} \\
\hline
Memahami konsep dasar dan definisi & 15 \\
\hline
Ketepatan analisis dan perhitungan & 20 \\
\hline
Penyajian jawaban yang sistematis & 15 \\
\hline
\end{tabular}

\vspace{0.5cm}

% Question 2 box
\noindent
\renewcommand{\arraystretch}{1.2}
\begin{tabular}{|>{\centering}m{2cm}|p{13cm}|}
\hline
\textbf{2} & \textbf{Pertanyaan:} Duis aute irure dolor in reprehenderit in voluptate velit esse cillum dolore eu fugiat nulla pariatur. Excepteur sint occaecat cupidatat non proident? \\
\hline
\end{tabular}

% Rubric table for question 2
\noindent
\renewcommand{\arraystretch}{1.2}
\begin{tabular}{|p{12cm}|p{3cm}|}
\hline
\textbf{Aspek / Konsep yang dinilai} & \textbf{Skor} \\
\hline
Identifikasi masalah dengan tepat & 10 \\
\hline
Penerapan metode yang sesuai & 10 \\
\hline
Kesimpulan yang logis & 5 \\
\hline
\end{tabular}

\newpage

% Question 3 box
\noindent
\renewcommand{\arraystretch}{1.2}
\begin{tabular}{|>{\centering}m{2cm}|p{13cm}|}
\hline
\textbf{3} & Sed ut perspiciatis unde omnis iste natus error sit voluptatem accusantium doloremque laudantium, totam rem aperiam, eaque ipsa quae ab illo inventore veritatis et quasi architecto beatae vitae dicta sunt explicabo? \\
\hline
\end{tabular}

% Rubric table for question 3
\noindent
\renewcommand{\arraystretch}{1.2}
\begin{tabular}{|p{12cm}|p{3cm}|}
\hline
\textbf{Aspek / Konsep yang dinilai} & \textbf{Skor} \\
\hline
Analisis mendalam dan kritis & 20 \\
\hline
Penggunaan referensi yang relevan & 10 \\
\hline
Struktur dan organisasi jawaban & 5 \\
\hline
\end{tabular}

\vspace{0.5cm}

% Question 4 box
\noindent
\renewcommand{\arraystretch}{1.2}
\begin{tabular}{|>{\centering}m{2cm}|p{13cm}|}
\hline
\textbf{4} & Nemo enim ipsam voluptatem quia voluptas sit aspernatur aut odit aut fugit, sed quia consequuntur magni dolores eos qui ratione voluptatem sequi nesciunt? \\
\hline
\end{tabular}

% Rubric table for question 4
\noindent
\renewcommand{\arraystretch}{1.2}
\begin{tabular}{|p{12cm}|p{3cm}|}
\hline
\textbf{Aspek / Konsep yang dinilai} & \textbf{Skor} \\
\hline
Pemahaman teori dan konsep & 8 \\
\hline
Kemampuan aplikasi praktis & 5 \\
\hline
Kreativitas dalam solusi & 2 \\
\hline
\end{tabular}

\vspace{1cm}

% OTORISASI heading
\begin{center}
\textbf{VALIDASI}
\end{center}

\vspace{0.5cm}

% Signature table (outside content border)
\noindent
\renewcommand{\arraystretch}{1.0}
\begin{tabular}{|>{\centering}p{4.8cm}|>{\centering}p{4.8cm}|>{\centering\arraybackslash}p{4.8cm}|}
\hline
Disahkan & Diperiksa & Dibuat \\
tanggal: ……………… & tanggal: ……………… & tanggal: ……………… \\
\hline
& & \\[3cm]
\hline
Koordinator Program Studi…….. & (Ketua KK…………………….) & (Dosen Penguji) \\
\hline
\end{tabular}

\end{document}